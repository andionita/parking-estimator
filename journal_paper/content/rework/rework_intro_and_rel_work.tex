\documentclass{article}

\newcommand{\cmmnt}[1]{\ignorespaces}

\begin{document}
	\cmmnt{TODO abstract}
	\section{Introduction}
	The parking problem and traffic congestion are commonplace in cities nowadays. While the number of cars continues to increase\cmmnt{ref}, studies show that about 30\% of the traffic in cities is caused by cars that are actively searching for parking\cmmnt{ref}. Often it is the lack of planning on behalf of cities that does not accommodate parking facilities proportional to building developments\cmmnt{search ref} which leads to double-parking, more accidents due to distracted drivers, more busy traffic, and, ultimately, a waste of fuel. As infrastructure solutions are not always optimal and take a long time to implement, there are other strategies to overcome these issues.
	
	Parking can be managed more efficiently if parking spaces would work on an allocation basis, with each driver reserving the spot of her choosing. Accounting for each individual spot and managing its reservations is, however, unrealistic and unsustainable at the moment. Instead, providing an overview of areas with free parking spaces would already be a big step forward. Suppose a driver would have access to a service that indicates city areas with free parking spaces at the time when she is arriving in the area. The system would take into account the usual parking levels in the various city areas depending on the day of the week and the time of day. Additional information such as current traffic, event data and weather would improve its estimations. Aware of this sort of parking information beforehand, the driver would pick a traveling path that is less busy through the city and spent significantly less time to find a parking spot that is limited to the area indicated.
	
	This work introduces an approach inspired by this vision that takes into account parking data and city infrastructure to produce estimations about parking spaces in various city areas. This paper extends our previous work\cmmnt{ref} by enriching the evaluation and exploring alternative assumptions of the original approach. Following the introduction, an overview of the research landscape in city parking is presented, before outlining the main assumptions behind our approach. The approach itself is broken down and described in detail, after which its evaluation is set up and carried out. Finally, further possible extensions are outlined and conclusions are drawn.

	\section{Related Work}
	

\end{document}