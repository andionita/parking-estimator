\documentclass{article}

\newcommand{\cmmnt}[1]{\ignorespaces}

\begin{document}
	\cmmnt{TODO abstract}
	\section{Introduction}
	The parking problem and traffic congestion are commonplace in cities nowadays. While the number of cars continues to increase\cmmnt{ref}, studies show that about 30\% of the traffic in cities is caused by cars that are actively searching for parking\cmmnt{ref}. Often it is the lack of planning on behalf of cities that does not accommodate parking facilities proportional to building developments\cmmnt{search ref}, which leads to double-parking, more accidents due to distracted drivers, more busy traffic, and, ultimately, a waste of fuel. As infrastructure solutions are not always optimal and take a long time to implement, there are other strategies to overcome these issues.
	
	Parking can be managed more efficiently if parking spaces would work on an allocation basis, with each driver reserving a spot of her choosing. Accounting for each individual spot and managing its reservations is, however, unrealistic and unsustainable at the moment. Merely providing an overview of areas with clusters of free parking spaces would already be a step forward. Suppose a driver would have access to a service that indicates city areas with free parking spaces at the time when she is arriving in the area. The system would take into account the usual parking levels in the various city areas depending on the day of the week and the time of day. Additional information such as current traffic, event data and weather would improve its estimations. Aware of this sort of parking information beforehand, the driver would pick a traveling path that is less busy through the city and spent significantly less time finding a parking spot that is limited to the area indicated.
	
	This work introduces an approach inspired by this vision that takes into account parking data and city infrastructure to produce estimations about parking spaces in various city areas. This paper extends our previous work\cmmnt{ref}, by enriching the evaluation and exploring alternative assumptions of the original approach. Following the introduction, an overview of the research landscape in city parking is presented, before outlining the main assumptions behind our approach. The approach itself is broken down and described in detail, after which its evaluation is set up and carried out. Finally, further possible extensions are outlined and conclusions are drawn.

	\section{Smart Parking Research}
	Improving the parking situation using sensor data has been a subject of research especially since 2000. In a 2017 survey \cmmnt{survey ref + phd ref}, Lin compiles an overview of the advances in smart parking by splitting the results into three categories: \textit{information collection}, \textit{system deployment}, and \textit{service dissemination}.
	
	\subsection{Information collection}
	Under information collection, the survey lists all techniques to acquire parking information. Static sensors are usually mounted around parking meters, in the ground or on nearby lamp posts. The dynamic sensors are usually managed through a wireless network infrastructure and are present inside cars, such as taxis, which travel through the city and collect roadside information on parking. In both cases, the piece of transmitted information is an occupancy change bit: either the car has left or arrived at a parking space. A number of types of sensors are usually being used in the process: infrared sensors, ultrasonic sensors, accelerators, optical sensors, inductive loops, piezoelectric sensors, cameras, acoustic sensors. Most of the time, the captured data requires post-processing in the form of image or audio recognition before arriving at the target occupancy information. Some of the captured information also raises privacy issues, as it contains sensitive data about the car and driver. Smartphones provide the means to collect data and can have a great impact through crowdsourcing, if the users are given incentives to enable the respective smartphone functions that automatically collect information. Conversely, the parking application may give drivers incentives to initiate the data transmission themselves and report the parking situation on site. 
	
	Data collection based on smartphone-integrated sensors has had numerous research publications, as it spared the authors to produce and mount special-purpose senors. Xu et al. \cmmnt{ref xu} makes real-time parking availability estimations based on a system that aggregates the data coming in from mobile phones. The system uses algorithms based on statistical weighted schemes and Kalman filters. Additionally, the authors create parking availability profiles based on historical data and using statistical algorithms. 
	
	Chen et al. \cmmnt{ref zchen} develop an Android application that finds a parking location at park-and-ride facilities by calculating the probability of parking availability and taking in consideration the shortest travel time. The authors employ fuzzy logic to model the uncertainty of parking availability, with the fuzzy membership function being linear. The authors proposed multiple criteria in finding the best parking location, such as train frequency, service quality, and parking-and-ride price. 
	
	PocketParker is a crowdsourcing system, proposed by Nandugudi et al. \cmmnt{ref nandugudi}, that uses smartphone data to predict parking availability. The system is used for parking lots. It requires no input from the user, it notices automatically when a user starts to drive or stops, i.e., departure and arrival events. Based on these two events, the system builds a probability distribution model that is used to answer queries about parking availability. PocketParker has proved robust to hidden parkers, i.e., parking vehicles that are not using the application. In the authors' simulation, it has reached 94\% rate for parking availability prediction with 105 users over 45 days. 
	
	Koster et al. \cmmnt{ref koster} propose a smartphone-based solution that recognizes when drivers arrive or leave parking spaces. A ``Bayesian approach'' and Hidden Markov Models (HMM) are used to model the parking spaces and respond to user queries for the next parking space. The HMM are based on gathered historical data. As answer to the user query is a parking space nearby and the probability of it being free at the respective time. The authors emphasize the non-intrusive nature of the solution, where drivers only have to minimally interact with their phones to get a recommended free parking space.
	
	\subsection{System Deployment}
	Regarding system deployment, the smart parking survey refers to the varieties of parking systems, looks into how well they scale, and touches upon the data analysis side. The parking system software is the interface between data sources and the users. Software systems are often in the form reservation systems, typically run by municipalities or private car parks. These systems may also provide guiding assistance in arriving to the desired parking lot or at the individual parking space. The vacancy prediction component informs the drivers about the availability of parking spaces at the destination, either in real-time or at a specific date and time.
	
	There are several research publications that include parking system worth mentioning. Rajabioun and Ioannou \cmmnt{ref rajabioun2013} introduce an information system for parking guidance that enables communication between vehicles and the infrastructure. It proposes a prediction algorithm that forecasts the availability for parking locations based on real-time parking information. It takes into account parameters such as parking duration, arrival time, destination, pricing, walking distance, parking capacity, rates of vehicles occupying and leaving parking spots, time restrictions, parking rules, events that disrupt parking availability, etc. Their algorithm uses a probabilistic density distribution model. The parking data was collected both from on-street parking meters and off-street garages in Los Angeles and San Francisco, USA. In a following paper, Rajabioun and Ioannou \cmmnt{ref rajabioun2015} propose a multivariate autoregressive model that considers the temporal and spatial correlations of parking availability when making predictions. 
	
	Tiedemann et al. \cmmnt{ref tiedemann} present the development of a prediction system that gives estimated occupancies for parking spaces. The occupancy data is collected online via roadside parking sensors and the prediction is realized using neural gas machine learning combined with data threads. The authors notice that some factors play a significant part in the predictions, such as holidays, weather and use the neural gas clustering to separate the data, before the data thread method is applied. 
	
	Richter et al. \cmmnt{ref richter} address the parking prediction problem with the focus on model storage in vehicles. The authors train models of various granularity that would predict parking availability based on the information contained: a one-day model per road segment, a three-day model per road segment, and a seven-day model per road segment. Additionally, models based on regions and time intervals computed by clustering are tried out. Hierarchical clustering with complete linkage is employed. The models are evaluated on street data from the SF\textit{park} project \cmmnt{ref sfpark_open_data}. The application of clustering before building the models shows a 99\% decrease of model storage space. The prediction success rate is at about 70\%. 
	
	With iParker, Kotb et al. \cmmnt{ref kotb} propose a system that handles parking reservations. It achieves resource allocation so that drivers will pay less for parking, while parking managers receive more resource utilization and hence reach higher revenue. The system is based on mixed-linear programming (MILP). The system uses dynamic resource allocation and pricing models to achieve its goal. At its evaluation, it was reported to cut the cost for drivers by 28\%, achieving a 21\% increase in resource utilization, and it increased the total revenue for parking management by 16\%. 
	
	Shin and Jun \cmmnt{ref shin} propose an algorithm for smart parking that assigns cars to parking facilities in the city. The criteria based on which the assigning is realized includes driving distance to the parking facility, walking distance from the parking facility to the destination, parking cost, and traffic congestion. The real-time data is collected from parking facilities and from sensors that are integrated in cruising cars. The data is transferred from the central server, where it is managed through a wired/wireless telecommunication network. The authors test their approach in Luxembourg City. The results of the simulations showed improved figures for average driving duration, average walking distance, parking failing rate, parking utilization rate, average standard deviation on the number of guided cars to each parking facility, average occupancy ratio of parking facility, and for the parking facility occupancy rate. 
	
	ParkNet, developed by Mathur et al. \cmmnt{ref mathur} is a system made up of vehicles that captures parking space information while driving. Every ParkNet vehicle is equipped with a GPS receiver and an ultrasonic sensor facing sideways. The latter determines whether it passes by parking spaces and whether they are occupied. The data is sent to a central server that aggregates it, in order to build parking space occupancy maps in real-time. The information is queried by clients that search for a free parking space. The system was evaluated in Highland Park, New Jersey and San Francisco on 500 miles road-side parking data and yield 95\% accurate parking maps and 90\% parking occupancy accuracy. The authors show that the system can further be improved if the sensors are fitted into taxicabs or city buses.
	
	\subsection{Data Dissemination}
	Under data dissemination, the survey addresses the capability of sharing parking information. This scenario occurs in decentralized parking systems, where the cars find out about free parking spaces in an area where other cars merely drive by and report the real-time situation. 
	
	A selected group of publication is driven by the information exchange approach. Caliskan et al. \cmmnt{ref caliskan} model the prediction of available parking spaces as a vehicular ad-hoc network (VANET). The network disseminates parking data in order to help with the estimation of future occupancy of parking lots. The pieces of disseminated data are timestamp, total capacity of parking lot, number of parking spaces that are currently occupied, the arrival rate, and the parking rate. The latter two are used in the modeling of continuous-time homogeneous Markov chains. The approach is otherwise based on queuing theory. 
	
	Klappenecker et al. \cmmnt{ref klappenecker} builds on the result of Caliskan and uses an improved version of continuous-time Markov chains for predicting availability of parking spaces. Predictions are communicated between cars in an ad-hoc network. The approach simplifies the computations of transitional probabilities inside a Markov chain model. The system applies to parking lots that are connected to the ad-hoc network. These communicate the number of occupied spaces, capacity, arrival and parking rate. 
	
	Also based on VANETs is Szczurek et al. \cmmnt{ref szczurek} work, which propose a novel approach that combines machine learning with the information disseminated in ad-hoc vehicular networks. The building blocks of the system are parking reports, which are issued by vehicles leaving a parking space and comprise a report identifier, a location, and a timestamp. The parking reports are being learned by a model, which then indicates whether a parking is available for a specific vehicle. A conditional relevance is used to determine whether a particular report is useful for a specific vehicle. This is modeled using a Naive Bayes method. A parking availability report R is labeled relevant by vehicle V, if the parking space referenced in R is available when V reaches it. Upon evaluation of the methods, the authors reported an improvement in parking discovery times for vehicles.
	
	\section{The SFpark project}
	A way of controlling parking occupancy is dynamic pricing. Parking prices are raised in areas that are almost fully occupied, whereas areas with low parking rates get assigned a lower price. A more advanced version adjusts the prices when enough demands received by the parking system would point to a future parking overload in the respective area.
	
\end{document}